% Created 2020-01-08 Wed 16:27
% Intended LaTeX compiler: pdflatex
\documentclass[11pt]{article}
\usepackage[utf8]{inputenc}
\usepackage[T1]{fontenc}
\usepackage{graphicx}
\usepackage{grffile}
\usepackage{longtable}
\usepackage{wrapfig}
\usepackage{rotating}
\usepackage[normalem]{ulem}
\usepackage{amsmath}
\usepackage{textcomp}
\usepackage{amssymb}
\usepackage{capt-of}
\usepackage{hyperref}
\date{\today}
\title{}
\hypersetup{
 pdfauthor={},
 pdftitle={},
 pdfkeywords={},
 pdfsubject={},
 pdfcreator={Emacs 26.2 (Org mode 9.1.9)}, 
 pdflang={English}}
\begin{document}

\tableofcontents

\section{Definition}
\label{sec:orgd9d4402}

Git is a distributed version control system. In contrast to distributed version control system(CVCS)


\section{commands}
\label{sec:orgaa74c8f}
\begin{itemize}
\item \texttt{git -{}-version}
\item \texttt{git config -{}-list}
\item \texttt{git status}
\item \texttt{git log}
\end{itemize}


\section{git ignore file}
\label{sec:orgd916ce1}
\begin{itemize}
\item \texttt{.DS\_STORE}
\item \texttt{.project}
\item \texttt{*.pyc}
\end{itemize}

\section{add files in staging are}
\label{sec:orga331c14}
\begin{itemize}
\item \texttt{git add -A}
\item \texttt{git reset} \emph{remove all files from staging area}
\end{itemize}

\section{commit}
\label{sec:orga2db7b4}
\begin{itemize}
\item \texttt{git commit -m "message"} \emph{commits staged files}
\end{itemize}

\section{clonong a remote repo in git}
\label{sec:org3461c02}
\begin{itemize}
\item \textasciitilde{}git clone <url> <where to clone>
\item \emph{ex} \texttt{git clone ../local-repo.git .} \emph{cloning a local repo inside a directory}
\end{itemize}

\subsection{view infornation about the remote repo}
\label{sec:org29048ad}
\begin{itemize}
\item \texttt{git remote -V}
\item \texttt{git branch -a}
\end{itemize}

\section{Pushing changes}
\label{sec:org94eceed}

\subsection{First commit like done before}
\label{sec:org835e924}
\begin{itemize}
\item \texttt{git diff}
\item 
\end{itemize}

\subsection{Then Push}
\label{sec:orgc4c81df}
\begin{itemize}
\item \texttt{git pull origin master}
\item \texttt{git push origin master}
\end{itemize}

\section{Common workflow while using git}
\label{sec:org0b6707f}

\subsection{Dont commit at the master branch, create a branch for a desired feature.}
\label{sec:org8af9063}
\begin{itemize}
\item \textasciitilde{}git branch <branch-name>
\item \textasciitilde{}git checkout <branch-name>
\item 
\end{itemize}

\subsection{After commit push branch to remote}
\label{sec:org486a396}
\begin{itemize}
\item \textasciitilde{}git push -u origin calc-divide \emph{-u tells that we wanna associate our local and remote branch}
\end{itemize}

\subsection{Merge a branch with master}
\label{sec:orgdb03692}
\begin{itemize}
\item \texttt{git checkout master}
\item \texttt{git pull origin master}
\item \texttt{git branch -{}-merged}
\item \texttt{git merge <branch-name>}
\item \texttt{git puch origin master}
\end{itemize}

\section{Errors and trouble shooting}
\label{sec:org25df07f}

\subsection{Bare and Non bare files problem}
\label{sec:orge978a2f}
\begin{verbatim}
Centros@Centros-PC MINGW64 /d/code/git/remote-repo (master)
$ git push origin master
Total 0 (delta 0), reused 0 (delta 0)
remote: error: refusing to update checked out branch: refs/heads/master
remote: error: By default, updating the current branch in a non-bare repository
remote: is denied, because it will make the index and work tree inconsistent
remote: with what you pushed, and will require 'git reset --hard' to match
remote: the work tree to HEAD.
remote:
remote: You can set the 'receive.denyCurrentBranch' configuration variable
remote: to 'ignore' or 'warn' in the remote repository to allow pushing into
remote: its current branch; however, this is not recommended unless you
remote: arranged to update its work tree to match what you pushed in some
remote: other way.
remote:
remote: To squelch this message and still keep the default behaviour, set
remote: 'receive.denyCurrentBranch' configuration variable to 'refuse'.
To D:/code/git/remote-repo/../local-repo/.git
 ! [remote rejected] master -> master (branch is currently checked out)
error: failed to push some refs to 'D:/code/git/remote-repo/../local-repo/.git
\end{verbatim}
\subsection{Solution Summary}
\label{sec:org7954250}

You cannot push to the one checked out branch of a repository because it would mess with the user of that repository in a way that will most probably end with \textbf{\textbf{loss of data and history}}. But you can push to any other branch of the same repository.

As bare repositories never have any branch checked out, you can always push to any branch of a bare repository.

\textbf{\textbf{There are multiple solutions, depending on your needs.}}

\begin{itemize}
\item Solution 1: Use a Bare Repostiory
\end{itemize}

As suggested, if on one machine, you don't need the working directory, you can move to a bare repository. To avoid messing with the repository, you can just clone it:

\begin{verbatim}
machine1$ cd ..
machine1$ mv repo repo.old
machine1$ git clone --bare repo.old repo

\end{verbatim}
Now you can push all you want to the same address as before.

\begin{itemize}
\item Solution 2: Push to a Non-Checked-Out Branch
\end{itemize}

But if you need to check out the code on your remote `<remote>`, then you can use a special branch to push. Let's say that in your local repository you have called your remote `origin` and you're on branch master. Then you could do

\textasciitilde{}    machine2\$ git push origin master:master+machine2\textasciitilde{}

Then you need to merge it when you're in the `origin` remote repo:

\textasciitilde{}    machine1\$ git merge master+machine2\textasciitilde{}

\begin{itemize}
\item Autopsy of the Problem
\end{itemize}

When a branch is checked out, committing will add a new commit with the current branch's head as its parent and move the branch's head to be that new commit.

So

A �� B
    ��
[HEAD,branch1]

becomes

A �� B �� C
	��
    [HEAD,branch1]

But if someone could push to that branch inbetween, the user would get itself in what git calls \textbf{\textbf{detached head}} mode:

A �� B �� X
    ��   ��
[HEAD] [branch1]

Now the user is not in branch1 anymore, without having explicitly asked to check out another branch. Worse, the user is now \textbf{\textbf{outside any branch}}, and any new commit will just be \textbf{\textbf{dangling}}:

      [HEAD]
	��
	C
      ��
A �� B �� X
	��
       [branch1]

Hypothetically, if at this point, the user checks out another branch, then this dangling commit becomes fair game for Git's \textbf{\textbf{garbage collector}}.
\end{document}
